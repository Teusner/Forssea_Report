L'implémentation d'un simulateur est un aspect important de la réalisation d'un projet robotique. En effet, il permet de gagner du temps de développement, mais aussi d'en réduire les coûts en testant le robot dans un environnement simulé. Il permet de placer les robots dans des situations contrôlées, de manière reproductible et sans risque de casse ou de panne matérielle lors d'essais.

Ce projet de fin d'étude concerne la réalisation d'un simulateur pour les \gls{ROV}s de \forssea{}. Il doit permettre de simuler le comportement de leurs robots dans un environnement sous-marin simulé afin de tester l'implémentation logicielle des nouvelles fonctionnalités sans avoir à organiser des campagnes d'essais en milieux naturels qui sont coûteux en temps et en ressources. Ceux-ci restent tout de même indispensables pour valider une nouvelle version du robot.
