L'implémentation d'un simulateur est un aspect important de la réalisation d'un projet robotique. En effet, il permet de gagner du temps de développement, mais aussi d'en réduire les coûts en testant le robot dans un environnement simulé. Il permet de placer les robots dans des situations contrôlées, de manière reproductible et sans risque de casse ou de panne matérielle lors d'essais.

Ce projet de fin d'étude concerne la réalisation d'un simulateur pour les \gls{ROV}s de \forssea{}. Il doit permettre de simuler le comportement de leurs robots dans un environnement sous-marin simulé afin de tester l'implémentation logicielle des nouvelles fonctionnalités sans avoir à organiser des campagnes d'essais en milieux naturels qui sont coûteux en temps et en ressources. Ces dernières restent tout de même indispensables pour valider une nouvelle version du robot.

Dans un premier temps, nous allons proposer un formalisme permettant de simuler le monde marin en décrivant les vagues, le courant, mais aussi des éléments qui vont influencer le comportement des robots, comme les ombilicaux. Ensuite nous allons voir comment simuler les robots en simulant finement chaque sous-composant, qui sont pour la plupart utilisés dans les deux \gls{ROV}s développés chez \forssea{} \argos{} et \atoll{}. Enfin nous pourrons analyser les résultats produits  par le simulateur et nous verrons qu'ils sont proches des données récupérées lors d'essais avec les \gls{ROV}s.