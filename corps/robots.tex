\chapter{Simulation des robots}
	
	\section{Introduction}
		Le simulateur doit permettre de simuler chacun des deux robots dans l'environnement défini dans le \textsc{Chapitre}~\ref{chapitre:environnement}. Pour ce faire, nous allons devoir décrire les paramètres mécaniques des robots \gls{Argos} et \gls{Atoll}, ainsi que leurs capteurs et actionneurs à simuler.

	\section{Simulation des composants}
		En remarquant que les deux robots embarquent un certain nombre d'éléments communs, il est possible de definir et de simuler les différents composants dans \gls{Gazebo}, afin d'être par la suite chargés dans la simulation des deux robots. La \textsc{table}~\ref{table:components} présente les différents composants à simuler, s'ils nécéssitent l'utilisation d'un \gls{Plugin}, d'une \gls{HardwareInterface} et indique les dépendances entre les robots et ces composants.

		\begin{table}[!htb]
			\centering
			\begin{adjustbox}{max width=\textwidth}
				\begin{tabular}{|l|l|c|c|c|c|}
					\hline
					Composant & Description & \gls{Argos} & \gls{Atoll} & \gls{HardwareInterface} & \gls{Gazebo} \gls{Plugin} \\
					\hline
					\gls{Argos} Frame & Chassis d'\gls{Argos} & \cmark & \xmark & \xmark & \xmark \\
					\hline
					\gls{Atoll} Frame & Chassis d'\gls{Atoll} & \xmark & \cmark & \xmark & \xmark \\
					\hline
					Electronic Pod & Boîtier electronique & \cmark & \cmark & \xmark & \xmark \\
					\hline
					\gls{Latch} & Crochet de levage & \xmark & \cmark & \cmark & \cmark \\
					\hline
					\gls{Navcam} & Caméra de navigation & \cmark & \cmark & \xmark & \cmark \\
					\hline
					\gls{Obscam} & Caméra d'observation & \cmark & \cmark & \xmark & \cmark \\
					\hline
					Rovins & Centrale Inertielle & \cmark & \cmark & \xmark & \cmark \\
					\hline
					Spotlight & Lumières étanches & \cmark & \cmark  & \cmark & \cmark \\
					\hline
					SPE75 Thruster & Propulseur & \cmark & \cmark & \cmark & \cmark \\
					\hline
					SS309 Tilt & Nacelle pour caméra & \cmark & \cmark & \cmark & \cmark \\
					\hline
				\end{tabular}}
			\end{adjustbox}
			\caption{Composants à simuler}
			\label{table:components}
		\end{table}

		Chaque composant a ainsi un \gls{Package} de description qui lui est associe nommé suivant la convention de nommage \gls{ROS} : \textit{component\_description}. Ce package contient tout le code nécéssaire à la description du composant, c'est à dire un modèle \gls{Gazebo} permettant de simuler le composant dans le logiciel de simulation, un fichier de lancement qui s'occupe de lancer le composant dans le simulateur, les \gls{Mesh CAO}. Le code source d'un \gls{Plugin} permettant de décrire le comportement du capteur ou de l'actionneur associé au composant se trouve dans le package \textit{component\_model\_plugin}.

	\section{Simulation d'Argos}

	\section{Simulation d'Atoll}