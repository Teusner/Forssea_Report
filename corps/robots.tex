\chapter{Simulation des robots}
	
	\section{Introduction}
		Le simulateur doit permettre de simuler chacun des deux robots dans l'environnement défini dans le \textsc{Chapitre}~\ref{chapitre:environnement}. Pour ce faire, nous allons devoir décrire les paramètres mécaniques des robots \gls{Argos} et \gls{Atoll}, ainsi que leurs capteurs et actionneurs à simuler.

		En remarquant que les deux robots embarquent un certain nombre d'éléments communs, il est possible de definir et de simuler les différents composants dans \gls{Gazebo}, afin d'être par la suite chargés dans la simulation des deux robots. La \textsc{table}~\ref{table:components} présente les différents composants à simuler et indique les dépendances entre les robots et ces composants.

		\begin{table}[!htb]
			\centering
			\begin{tabular}{|l|l|c|c|}
				\hline
				Composant & Description & \gls{Argos} & \gls{Atoll} \\
				\hline
				Argos Frame & Chassis d'\gls{Argos} & \cmark & \xmark \\
				\hline
				Atoll Frame & Chassis d'\gls{Atoll} & \xmark & \cmark \\
				\hline
				Electronic Pod & Boîtier electronique & \cmark & \cmark \\
				\hline
				Latch & Crochet de levage & \xmark & \cmark\\
				\hline
				\gls{Navcam} & Caméra de navigation & \cmark & \cmark\\
				\hline
				\gls{Obscam} & Caméra d'observation & \cmark & \cmark\\
				\hline
				Rovins & Centrale Inertielle & \cmark & \cmark\\
				\hline
				Spotlight & Lumières étanches & \cmark & \cmark \\
				\hline
				SPE75 Thruster & Propulseur & \cmark & \cmark \\
				\hline
				SS309 Tilt & Nacelle orientable pour caméra & \cmark & \cmark \\
				\hline
			\end{tabular}
			\caption{Composants à simuler}
			\label{table:components}
		\end{table}

		Chaque composant a ainsi un \gle{Package} de description nommé suivant la convention de nommage \gls{ROS} : \textit{nom_du_package_description}. Ce package contient 

	\section{Simulation des sous-composants}
		\subsection{\gls{Argos} Frame}


	\section{Simulation d'Argos}

	\section{Simulation d'Atoll}