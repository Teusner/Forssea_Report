\chapter{Simulation de l'environnement}
	
	\section{Introduction}

		La simulation de l'environnement est une partie importante du simulateur car c'est ce qui va le plus influencer le comportement des \gls{ROV}s dans leur milieu. Elle doit être très proche des conditions réelles afin d'avoir des résultats exploitables. Il faut donc établir les différents élements de l'environnements qui vont interagir avec les \gls{ROV}s afin de les intégrer dans le simulateur. Il y a principalement le milieu marin qui va ajouter une flottabilité au robots, mais aussi la présence de courants et enfin l'ombilical qui relie les \gls{ROV}s au bateau afin de l'alimenter en énergie mais aussi d'avoir un retour d'informations.

	\section{Simulation d'environnements marins}

	\section{Simulation des courants marins}

	\section{Simulation d'ombilicaux}

		\subsection{Etat de l'art}

\subsection{Formalisme}

\subsection{Initialisation}

\subsection{Implémentation}

\subsection{Suivi d'angles normalisés}
    Un problème avec la représentation numérique de l'orientation des solides est qu'elle est souvent normalisée, et les valeurs sont ainsi ramenées dans l'intervalle $[-\pi; \pi]$. On ne peut donc pas avoir l'orientation absolue, c'est à dire l'orientation d'un solide en prenant en compte les eventuels tours qu'il aurait pu faire sur lui-même.

    Pour résoudre ce problème, l'\textsc{Algorithme}~\ref{algo:suivi_angle} de suivi d'angles normalisés a été implémenté. Il prends en paramètres l'angle normalisé ainsi que l'angle précédemment calculé, et il retourne la valeur de l'angle absolu. L'idée de fournir l'angle précédent est de pouvoir retourner le nouvel angle qui se trouve dans le même quadrant et aussi de pouvoir suivre les sauts d'angles. Ainsi on peut suivre l'orientation absolue de solides en rotation dans l'espace, en ne fournissant que des orientations relatives ramenées dans l'intervalle $[-\pi; \pi]$, et en gardant en mémoire la précédente orientation calculée.
    
    \begin{algorithm}[!htb]
        \SetKwInOut{Input}{Entrées}
        \SetKwInOut{Output}{Sorties}
        \Entree{$angle\_normalise$, $angle\_absolu$}
        \Sortie{$angle\_absolu$}
        \Deb{
            $offset \leftarrow (angle\_absolu - angle\_normalise + \pi ) \pmod{2\pi}$ \\
            $angle\_absolu \leftarrow angle\_normalise + 2\pi \cdot offset$ \\
        }
        \Retour{$angle\_absolu$}

        \caption{Suivi d'angle} 
        \label{algo:suivi_angle}
    \end{algorithm}

    La \textsc{Figure}~\ref{fig:suivi_angle} présente les résultats de l'\textsc{Algorithme}~\ref{algo:suivi_angle} avec une angle variant dans l'intervalle $[-3\pi; 3\pi]$. On a dans la première sous-figure l'angle réel et l'angle ramené dans l'intervalle $[-\pi; \pi]$. On est capable de suivre l'évolution de l'angle et de supprimer les sauts afin de retrouver l'angle absolu présenté dans la deuxième sous-figure.

    \begin{figure}[!htb]
        \centering
        \includegraphics[width=0.5\textwidth]{suivi_angle.png}
        \caption{Suivi d'angle}
        \label{fig:suivi_angle}
    \end{figure}


\subsection{Resultats}
	