\chapter{Annexes}
\label{annexe:gantt}

	\section{Simplification en eaux profondes}

		Nous allons ici justifier la simplification faite en eaux profondes dans la \textsc{Section}~\ref{sec:simplification_eaux_profondes}. Nous avons en effet pu proposer une approximation en \textsc{Equation}~\ref{eqn:simplification_eaux_profondes} qui permet de simplifier les équations décrivant la dynamique du milieu marin.

		\begin{eqnarray}
			\frac{cosh(k_m(z+H))}{sinh(k_mH)} = \left( \frac{e^{z+H} + e^{-z+H}}{e^{H}-e^{-H}} \right)^{k_m}  \xrightarrow[H \rightarrow + \infty]{}   \left( \frac{e^{z+H}}{e^{H}} \right)^{k_m} = e^{k_m z}
		\end{eqnarray}

		Pour la simplification de l'\textsc{Equation}~\ref{equation:dispersion} en l'\textsc{Equation}~\ref{equation:deep_dispersion} de dispersion des vagues, cela se justifie par le fait que :

		\begin{eqnarray}
			tanh(k_m H) \xrightarrow[H \rightarrow + \infty]{} 1
		\end{eqnarray}

	\clearpage

	\begin{figure}[H]
		\centering
		\rotatebox{90}{
			\includegraphics[width=21cm]{gantt_before.pdf}
		}
        \label{fig:gantt_before}
        \caption{Diagramme de Gantt prévisionnel du projet}
	\end{figure}

	\begin{figure}[H]
		\centering
		\rotatebox{90}{
			\includegraphics[width=21cm]{gantt_after.pdf}
		}
        \label{fig:gantt_after}
        \caption{Diagramme de Gantt final du projet}
	\end{figure}