\chapter{Gestion de projet}
	
	\section{Introduction}
		Ce rapport présente le travail effectué durant ce projet de fin d'études. Le format est particulier dans la mesure où j'ai réalisé la dernière année de ma formation en contrat de professionnalisation avec l'entreprise Forssea Robotics. J'ai donc pu travailler pendant un an avec eux, en commencant par 6 mois durant lesquels j'ai fini ma formation au sein de l'\gls{ENSTAB}, puis 6 mois durant lesquels j'ai pu travailler à temps plein sur mon projet de fin d'études. Cela présente l'avantage d'avoir à sa charge la gestion d'un projet plus complet, puisque le temps le permet, comparé à un projet de fin d'étude classique dans lequel il est souvent confié au stagiaire la réalisation d'une tâche d'un projet.

	\section{Méthodologie}
		Les \textit{Methodes Agiles} forment un ensemble de méthodes facilitant la gestion de projet, qui sont particulièrement adaptées au monde du développement logiciel. Il permet de réaliser une plannification adaptative, un développement évolutif et une amélioration continue du produit. Elles s'opposent aux méthodes plus traditionnelles, comme les méthodes séquentielles de type \textit{Cycle en V}, qui s'adaptent très mal à ce type de produit. Elles permettent aussi d'avoir un produit utilisable par le client, avec des fonctionnalités qui évoluent au cours du développement.

		La \textit{Methode Scrum} est un cadre hérité de la \textit{Méthode Agile} permettant lui aussi de gérer le développement d'un produit. Il se distingue des \textit{Méthodes Agiles} dans la mesure où ce n'est pas seulement un ensemble de concepts, mais plutôt ici un ensemble de règles à suivre pour gérer correctement un projet. La \textit{Méthode Scrum} nécéssite la désignation de :

		\begin{itemize}
			\item Un \textit{Scrum Master} : garant de l'application de la \textit{Méthode Scrum},
			\item Un \textit{Product Owner} : garant des attentes du client au sein du projet,
			\item Une équipe de développement : réalisant le produit.
		\end{itemize}

		Les temps forts de la \textit{Méthode Scrum} sont :

		\begin{itemize}
			\item La plannification de sprint : réunion durant laquelle les nouvelles fonctionnalités à ajouter au produit sont séléctionnées en accord avec le \textit{Product Owner},
			\item La revue de sprint : réunion se déroulant en général après le sprint et durant laquelle les nouvelles foncitonnalités implémentées durant le sprint sont présentées au \textit{Product Owner} et au client, et le prochain sprint est préparé,
			\item La retrospective de sprint : réunion dans laquelle l'équipe analyse sa propre gestion de projet en terme d'efficacité, de qualité, de productivité,
			\item La mêlée quotidienne : réunion dans laquelle l'équipe de développement expose ce qu'elle a réalisé, ce qu'elle va réaliser et les problèmes qui ont été rencontrés depuis la dernière mêlée.
		\end{itemize}
	
		

	\section{Outil de gestion de projet}
		Jira -> methode scrum
		Github -> Versionning
	