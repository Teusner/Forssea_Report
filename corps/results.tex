\chapter{Resultats}
	
	\section{Introduction}

		Nous allons à présent tester le bon fonctionnement et analyser les résultats produits par le simulateur.
		
		Dans un premier temps, nous allons vérifier le respect des exigences présentées dans le \textsc{Chapitre}~\ref{chapitre:systeme}, ainsi que le comportement des robots et de l'environnement développés dans les \textsc{Chapitre}~\ref{chapitre:environnement} et \textsc{Chapitre}~\ref{chapitre:robots}.

		Ensuite, nous nous pencherons sur la comparaison du simulateur avec des expérimentations faites sur le terrain, afin de déterminer si les résultats de simulations sont proches du comportement réel des robots, et si ce système permet de réduire les temps de développements en en réduisant le nombre d'essais en conditions réelles à réaliser pour valider de nouvelles fonctionnalités.

	\section{Analyse du simulateur}

	\section{Comparaison avec les robots réels}

		Ecarts (statique/dynamique) lors d'essais en bassins ou réels
