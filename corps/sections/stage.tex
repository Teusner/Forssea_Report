\subsection{Cadre professionnel}
    Ce projet de fin d'étude s'inscrit dans le cadre d'un contrat de professionnalisation réalisé chez Forssea Robotics, entre octobre 2020 et octobre 2021. Il m'a permis d'effectuer un premier pas dans le monde industriel tout en finissant ma formation au sein de l'\gls{ENSTAB}. J'ai donc pu avoir à ma charge un sujet plus complet étant donnée de la durée de ce stage. J'ai aussi pu avoir plus de responsabilités dans la gestion de ce projet dans la mesure ou j'ai intégré l'équipe robotique de Forssea Robotics pendant une année complète dont six mois à temps plein.

\subsection{Présentation du sujet}
    Dans les phases de développements des robots, Forssea Robotics doit faire face à de nombreuses problématiques, dont celle de tester les algorithmes implémentés sur les \gls{ROV}s. Pour cela l'entreprise réalise régulièrement des tests et conditions réelles, mais ce sont des essais coûteux et qui prennent beaucoup de temps à planifier et à organiser. Afin de réduire le nombre d'essais nécessaires, il serait préférable de développer un moyen de test rapide et fiable des robots dans leur environnement. Cela permettrait à la fois d'augmenter les performances des robots et de diminuer les coûts et les temps de développements.

\subsection{Objectifs du projet}
    L'objectif de ce stage est donc de construire un environnement de simulation sous-marin pour Forssea Robotics. Il devra permettre de simuler au mieux le comportement des robots dans leur milieu, mais aussi de permettre de tester leurs algorithmes de navigations autonomes. Cet environnement de simulation permet évidemment de tester plus rapidement les implémentations, de construire des situations difficilement trouvables en conditions réelles, mais aussi d'avoir une répétabilité des situations. En outre, il ne se substitue pas à la réalisation de tests en conditions réelles.

   