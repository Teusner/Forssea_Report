\subsection{Sujet}
    Dans les phases de développements des robots, Forssea Robotics doit faire face à de nombreuses probélmatiques, dont celle de tester ses algorithmes. Pour cela l'entreprise réalise régulièrement des tests et conditions réelles, mais ce sont des essais qui son coûteux et qui prennent beaucoup de temps à plannifier et à organiser. Afin de réduire les temps de développements et les essais, il serait préférable de trouver un moyen de test rapide et fiable des robots dans leur environnement afin d'augmenter les performances des robots et de diminuer les coûts et les temps de développements.

\subsection{Objectifs}
    L'objectif de ce stage est de construire un environnement de simulation sous-marin pour Forssea Robotics. Il devra permettre de simuler au mieux le comportement des robots dans leur milieu, mais aussi de permettre de tester leurs algorithmes de navigations autonomes. Cet environnement de simulation permet évidemment de tester plus rapidement les implémentations, de construire des situations difficilements trouvables en conditions réelles, mais aussi d'avoir une répétabilité des situations. En outre, il ne se substitue pas à la réalisation de tests en conditions réelles.

    Ce projet de fin d'étude s'inscrit dans le cadre d'un contrat de professionalisation réalisé chez Forssea Robotics. Il m'a permis d'effectuer un premier pas dans le monde industriel tout en finissant ma formation au sein de l'\gls{ENSTAB}. J'ai donc pu avoir un sujet plus complet