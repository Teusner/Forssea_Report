\chapter{Conception et analyse du système}

    \section{Introduction}

        La simulation de milieux marins est quelque chose de connu dans le domaine de la robotique, et il en existe aujourd'hui déjà de nombreux~\cite{Manhaes_2016, bingham19toward, MARS, Rock}. Ces simulateurs proposent tous un modèle de flottabilité pour les objets à simuler, ce qui permet de pouvoir définir à la fois des bateaux et des robots sous-marins. Cela permet donc de pouvoir simuler le comportement des \gls{ROV}s dans l'eau. Ils proposent ensuite tous leur lot de spécificités qui permettent d'avoir des éléments supplémentaires dans notre simulation.
        
        \textsc{UUV Simulator}~\cite{Manhaes_2016} est probablement le plus connu d'entre eux. C'est un simulateur de \gls{ROV} comportant un certain nombre d'éléments d'environnements simulés, comme des mondes marins, du courant ou des ombilicaux par exemple. Il propose aussi la description d'un bon nombre de robots sous-marins commercialisés, et la communauté active partage les nouveaux modèles régulièrement.
        
        \textsc{Mars}~\cite{MARS} est un simulateur spécifique à des robot baptisés \textsc{Mars}. Il permet de supporter plusieurs \gls{ROV}s et peut être commandé par \gls{ROS} ou bien en utilisant des sockets TCP.
        
        \textsc{Rock-Gazebo}~\cite{Rock} est un projet d'intégration de simulateur basé sur le moteur physique \gls{Gazebo} et sur le framework\footnote{Infrastructure logicielle facilitant le développement logiciel} \textsc{Robot Construction Kit}.
        
        Le simulateur \textsc{VRX}~\cite{bingham19toward} est un projet open-source de robotique marine dont un simulateur a été implémenté sur la base de \gls{Gazebo}. Il propose un ensemble d'environnement, de modèles et de plugin permettant la simulation de missions de vaisseaux de surface, avec notamment la possibilité de prendre en compte la présence de vagues et de vent à la surface.

        Ces simulateurs proposent tous des spécificités différentes et intéressantes. Cependant, notre application de simulation de \gls{ROV}s pour \textsc{Forssea Robotics} est assez contrainte. En effet, l'existance de composants spécifiques, comme la simulation du \gls{Latch} ainsi que la \gls{frameLBL} rendent l'adaptation de simulateurs déjà existants trop difficile. C'est pourquoi il est nécéssaire de réaliser notre propre simulateur.

    \section{Spécifications fonctionnelles}
        \label{sec:spec_fonc}

        Le système à concevoir doit permettre de simuler \gls{ROV}s de la société \textsc{Forssea Robotics} dans un environnement sous-marin. Le simulateur doit respecter un certain nombre de spécifications fonctionnelles qui délimitent les exigences requises par le système. La \textsc{Figure}~\ref{figure:pieuvre} complétée par la \textsc{Table}~\ref{table:specs} présentent les exigences du système.

        \begin{figure}[!htb]
            \centering
            \begin{tikzpicture}[scale=0.7, every node/.style={ellipse, align=center, scale=0.7}]
                \node[draw,minimum height=40pt, minimum width=100pt, fill=Lavender!80!pink!100] (S) at (0, 0) {Simulateur};
                \node[draw,minimum height=40pt, minimum width=100pt, fill=cyan!60!pink!70] (C0) at (25:5.5cm) {Robot};
                \node[draw,minimum height=40pt, minimum width=100pt, fill=cyan!60!pink!70] (C1) at (155:5.5cm) {Environnement};
                \node[draw,minimum height=40pt, minimum width=100pt, fill=cyan!60!pink!70] (C2) at (205:5.5cm) {Communication};
                \node[draw,minimum height=40pt, minimum width=100pt, fill=cyan!60!pink!70] (C3) at (335:5.5cm) {Interface};
                
                \draw[blue!75, line width=0.8mm] (C0) -- (S) node[midway, fill=white]{FP1};
                \draw[blue!75, line width=0.8mm] (C1) -- (S) node[midway, fill=white]{FP2};
                \draw[orange!80, line width=0.8mm] (C2) -- (S) node[midway, fill=white]{FC1};
                \draw[orange!80, line width=0.8mm] (C3) -- (S) node[midway, fill=white]{FC2};
            \end{tikzpicture}
            \caption{Diagramme pieuvre du simulateur}
            \label{figure:pieuvre}
        \end{figure}

        \begin{table}[!htb]
            \centering
            \begin{adjustbox}{max width=\textwidth}
                \begin{tabularx}{\textwidth}{|lX|}
                    \hline
                    \cellcolor{gray!25}\textbf{id} & \cellcolor{gray!25} \textbf{Fonction} \\
                    \hline \hline
                    \cellcolor{blue!30}\textbf{FP1}&\cellcolor{blue!25} Le système doit avoir un comportement proche du robot réel. \\
                    \hline
                    \cellcolor{gray!10}FP1.1& Le système doit avoir des dimensions semblables au robot réel. \\
                    \hline
                    \cellcolor{gray!10}FP1.2& Le système doit avoir des masses et des inerties semblables au robot réel. \\
                    \hline
                    \cellcolor{gray!10}FP1.3& Le système doit avoir les mêmes degrès de liberté que le robot réel. \\
                    \hline
                    \cellcolor{gray!10}FP1.5& Le système doit avoir les mêmes actionneurs que le robot réel. \\
                    \hline
                    \cellcolor{gray!10}FP1.4& Le système doit avoir les mêmes capteurs que le robot réel. \\
            
                    \hline \hline
            
                    \cellcolor{blue!30}\textbf{FP2}&\cellcolor{blue!25} Le système doit simuler l'environnement du robot \\
                    \hline
                    \cellcolor{gray!10}FP2.1& Le système doit simuler l'environnement sous-marin. \\
                    \hline
                    \cellcolor{gray!10}FP2.2& Le système doit simuler la présence de courants marins. \\
                    \hline
                    \cellcolor{gray!10}FP2.2& Le système doit simuler les ombilicaux reliant le \gls{ROV} au bateau. \\
                    \hline
            
                    \hline \hline
            
                    \cellcolor{orange!40}\textbf{FC1} &\cellcolor{orange!30} Le système doit communiquer son état à l'utilisateur. \\
                    \hline
                    \cellcolor{gray!10}FC1.1& Le système doit communiquer l'état du robot à l'utilisateur. \\
                    \hline
                    \cellcolor{gray!10}FC1.2& Le système doit communiquer l'état de l'environnement à l'utilisateur.\\
                    \hline
            
                    \hline \hline
            
                    \cellcolor{orange!40}\textbf{FC2}&\cellcolor{orange!30} Le système doit être interfaceable avec le reste de l'implémentation logicielle. \\
                    \hline
                    \cellcolor{gray!10}FC2.1& Le système doit pouvoir évoluer dans un monde vide. \\
                    \hline
                    \cellcolor{gray!10}FC2.2& Le système doit pouvoir évoluer dans une vigne ayant une largeur paramétrable. \\
                    \hline
                    \cellcolor{gray!10}FC2.3& Le système doit pouvoir entrer en collision avec les objets solides de l'environnement de simulation. \\
                    \hline
                \end{tabularx}
            \end{adjustbox}
            \caption{Spécifications fonctionnelles du simulateur}
            \label{table:specs}
        \end{table}    

    \section{Spécifications techniques}

        Pour remplir les différentes spécifications fonctionnelles présentées dans la \textsc{Section}~\ref{sec:spec_fonc}, nous allons définir un certain nombre de spécifications techniques. Ces dernières vont proposer les outils et les différents critères permettant de remplir les exigences attendues.

        \subsection{Logiciel de simulation}

            L'étude de l'existant nous aura au moins permis de se rendre compte que \gls{Gazebo}~\cite{Koenig-gazebo} est largement utilisé dans le domaine de la simulation sous-marine, mais aussi plus largement dans la simulation robotique. \gls{Gazebo} est un logiciel de simulation multi-physique open-source\footnote{Code communautaire ouvert libre de redistribution et d'accès}. Il est aujourd'hui développé par la communauté \gls{OpenRobotics}. Il présente l'avantage de pouvoir communiquer avec \gls{ROS}, ce qui est une exigence importante de notre simulateur, car il doit permettre de tester l'implémentation logicielle des robots. La description des robots se fait en utilisant le language \gls{SDF}\footnote{\url{http://sdformat.org/}}, et il est possible de décrire le comportement complexe d'un composant en implémentant un \gls{Plugin}.

            L'utilisation de ce logiciel nous permet donc de remplir la fonction principale \textbf{FP1}, dans la mesure où il est possible de décrire tout les composants des \gls{ROV}s avec les trois couches visuelles, de collision et d'inertie, mais aussi les capteurs et les actionneurs dans l'environnement de simulation de \gls{Gazebo}. Il est aussi possible de décrire le comportement de l'environnement du robot dans ce simulateur, ce qui permet de remplir la fonction principale \textbf{FP2}.

        \subsection{Ignition Libraries}

            \textit{Ignition Libraries}\footnote{\url{https://ignitionrobotics.org/libs}} est un ensemble de libraries permettant de faciliter le développement robotique. Ces librairies sont mises à disposition par \gls{OpenRobotics}. Deux libraries permettent de remplir la fonction contrainte \textbf{FC1} : \textit{Ignition-Transport} et \textit{Ignition-Msgs}. Elles permettent de faire communiquer deux parties de codes ensemble grâce à un système de messagerie basée sur l'utilisation de la bilbiothèque \textit{ProtocolBuffer} développée par \textit{Google}\footnote{\url{https://developers.google.com/protocol-buffers}}. C'est le même système de communication qui est utilisé par \gls{ROS} et \gls{Gazebo}. Il présente l'avantage de diviser le code en n\oe uds, mais aussi de pouvoir récupérer l'état du simulateur par la lecture des messages transitants sur le réseau \textit{Ignition-Transport}.

        \subsection{ROS2 Control}

            Le simulateur doit permettre de s'interfacer avec le reste de l'implémentation logicielle, il est nécéssaire de définir une architecture logicielle permettant de communiquer à la fois avec les \gls{ROV}s réels et simulés. \gls{ROS2Control}~\cite{ros_control} est un framework permettant de faire communiquer des contrôleurs et des drivers ensemble. Le principal avantage est qu'il est temps-réel, c'est-à-dire que l'on peut changer de manière consistante de contrôleur en foncitonnement, sans temps mort durant lesquels le robot ne serait pas commandé.

            \begin{figure}[!htb]
                \centering
                \includegraphics{imgs/ros2_control.pdf}
                \caption{Architecture logicielle avec \gls{ROS2Control}}
                \label{fig:ros2_control}
            \end{figure}

            La \textsc{Figure}~\ref{fig:ros2_control} nous montre l'architecture logicielle de \gls{ROS2Control}. Ce framework va instancier un \gls{ControllerManager} qui va s'occuper de charger les contrôleurs qui seront utilisés dans le robot, des \gls{HardwareInterface} qui sont les interfaces qui vont communiquer avec les composants et un \gls{RessourceManager} qui va faire transiter les données entre les contrôleurs et les composants. L'\gls{HardwareInterface} peut avoir pour rôle de communiquer avec les composants réels, où avec les composants simulés. L'idée est de créer une \gls{HardwareInterface} proposant les mêmes interfaces que celle implémentée pour les composants réels. Ainsi on peut utiliser les mêmes algorithmes de contrôle de manière totalement transparente pour commander le robot réel ou le simulateur.

        \subsection{Architecture Logicielle}

            En rassemblant ce qui a pu être présenté dans cette \textsc{Section}, on peut proposer une architecture logicielle pour ce simulateur présentée sur la \textsc{Figure}~\ref{fig:architecture_logicielle}.
            
            \begin{figure}[!htb]
                \centering
                \includegraphics[]{imgs/architecture_logicielle.pdf}
                \caption{Architecture logicielle du simulateur}
                \label{fig:architecture_logicielle}
            \end{figure}