\newglossaryentry{ENSTAB}{type=glo,
	text={\textsc{ENSTA} Bretagne},
	name={\textsc{ENSTA} Bretagne},
	description={Ecole Nationale Supérieure de Techniques Avancées Bretagne}}

\newglossaryentry{ROV}{type=glo,
	text={\textsc{ROV}},
	name={\textsc{ROV}},
	description={Remotely Operated Vehicles, véhicule sous-marin téléguidé}}

\newglossaryentry{Argos}{type=glo,
	text={\textsc{Argos}},
	name={\textsc{Argos}},
	description={\gls{ROV} commercialisé par Forssea Robotics}}

\newglossaryentry{Atoll}{type=glo,
	text={\textsc{Atoll}},
	name={\textsc{Atoll}},
	description={\gls{ROV} commercialisé par Forssea Robotics}}

\newglossaryentry{Navcam}{type=glo,
	text={\textsc{Navcam}},
	name={\textsc{Navcam}},
	description={camera de navigation réalisant un traitement temps-réel basé sur des algorithmes d'intelligence artificielle}}

\newglossaryentry{Obscam}{type=glo,
	text={\textsc{Obscam}},
	name={\textsc{Obscam}},
	description={camera d'observation}}

\newglossaryentry{frameLBL}{type=glo,
	text={\textsc{Frame lbl}},
	name={\textsc{Frame lbl}},
	description={Cadre portant des balise Long Baseline permettant la localisation sous-marine}}

\newglossaryentry{Gazebo}{type=glo,
	text={\textsc{Gazebo}},
	name={\textsc{Gazebo}},
	description={Logiciel de simulation multi-physique dédié à la robotique : \url{http://gazebosim.org/}}}

\newglossaryentry{ROS}{type=glo,
		text={\textsc{ROS}},
		name={\textsc{ROS}},
		description={Middleware dédié à la robotique : \url{https://www.ros.org/}}}

\newglossaryentry{OpenRobotics}{type=glo,
	text={\textsc{Open Robotics}},
	name={\textsc{Open Robotics}},
	description={Communauté de programmeur fournissant des bibliothèques visant à faciliter le développement robotique comme \gls{ROS} et \gls{Gazebo} : \url{https://www.openrobotics.org/}}}

\newglossaryentry{SDF}{type=glo,
	text={\textsc{SDF}},
	name={\textsc{SDF}},
	description={(Simulation Description Format) Format de fichier permettant la description de simulations}}

\newglossaryentry{Plugin}{type=glo,
	text={\textsc{Plugin}},
	name={\textsc{Plugin}},
	description={Module d'extension permettant d'ajouter des fonctionnalités à un logiciel}}

\newglossaryentry{ROS2Control}{type=glo,
	text={\textsc{ROS2 Control}},
	name={\textsc{ROS2 Control}},
	description={Framework permettant de faire le lien entre des contrôleurs et des composants.}}

\newglossaryentry{ControllerManager}{type=glo,
	text={\textsc{Controller Manager}},
	name={\textsc{Controller Manager}},
	description={Gestionnaire de contrôleurs utilisé dans le framework \gls{ROS2Control}.}}

\newglossaryentry{RessourceManager}{type=glo,
	text={\textsc{Ressource Manager}},
	name={\textsc{Ressource Manager}},
	description={Gestionnaire de ressources utilisé dans le framework \gls{ROS2Control}.}}

\newglossaryentry{HardwareInterface}{type=glo,
	text={\textsc{Hardware Interface}},
	name={\textsc{Hardware Interface}},
	description={Composant peremttant de faire le lien entre l'implémentation logicielle et le composant réel ou simulé dans le Framework ROS2 Control.}}

\newglossaryentry{Package}{type=glo,
	text={\textsc{Package}},
	name={\textsc{Package}},
	description={Paquet formant un ensemble cohérent de code et pouvant contenir des bibliothèques, des exécutables, des scripts ou d'autres objets}}

\newglossaryentry{Cpp}{type=glo,
	text={\textsc{C++}},
	name={\textsc{C++}},
	description={Language de programmation}}

\newglossaryentry{GSL}{type=glo,
	text={\textsc{GSL}},
	name={\textsc{GSL}},
	description={GNU Scientific Library, une librairie \gls{Cpp} permettant de résoudre des problèmes mathématiques}}

\newglossaryentry{Mesh CAO}{type=glo,
	text={\textsc{Mesh Cao}},
	name={\textsc{Mesh Cao}},
	description={Fichier de maillage exporté d'un logiciel de Conception Assistée par Ordinateur permettant de représenter visuellement un objet simulé}}

\newglossaryentry{Latch}{type=glo,
	text={\textsc{Latch}},
	name={\textsc{Latch}},
	description={Crochet mécanique commandable permettant d'aggripper des \gls{frameLBL}}}

\newglossaryentry{Link}{type=glo,
	text={\textsc{Link}},
	name={\textsc{Link}},
	description={Solide simulé dans \gls{Gazebo}}}

\newglossaryentry{Joint}{type=glo,
	text={\textsc{Joint}},
	name={\textsc{Joint}},
	description={Liaison entre deux \gls{Link} dans \gls{Gazebo}}}