\chapter{Conclusion}
\chaptermark{Conclusion}
\label{chapitre:conclusion}
	
	\section{Conclusion technique}

		La réalisation de ce simulateur est un point très important dans le domaine du développement robotique. Je suis conscient des gains apportés par un tel système en termes de coûts de développements, car les essais en robotique restent très onéreux, la puissance de calculs de nos ordinateurs est aujourd'hui largement suffisante, et les logiciels de simulations de plus en plus complets et puissants. 
		
		Un exemple marquant est que durant mon projet de fin d'études, une semaine d'essai a été organisée afin de tester de nouvelles fonctionnalités d'\argos{}. Cependant, pour cause de mauvais temps le bateau n'a pas pu quitter le port. Cela a nécessairement mobilisé du matériel et des ressources humaines pour tester des fonctionnalités qui auraient pu être testée en simulation dans un premier temps, avant de réaliser une campagne d'essais finale permettant de valider la nouvelle version complète du robot.
	
	\section{Conclusion du projet}

		Ce simulateur a été réalisé durant un contrat de professionnalisation avec \forssea{} et dans la continuité de mon stage de deuxième année réalisé aussi sur des aspects de simulation en développement robotique sur \gazebo{} dans la jeune pousse \textit{Exxact Robotics}. J'ai donc pu me spécialiser cette dernière année dans la simulation robotique et notamment sur l'utilisation du simulateur \gazebo{} qui est très permissif et dans lequel j'ai pu beaucoup apprendre.

		En termes de gestion de projets, il a été très intéressant d'avoir à ma charge la gestion de l'intégralité du projet simulation. Je trouve que c'est une bonne introduction à mon futur parcours professionnel et l'encadrement de mon maître de stage m'a permis de mener à bien ce projet.

		La conception de ce système m'a amené à formaliser beaucoup d'aspects du simulateur, et le côté start-up de \textsc{Forssea Robotics} fait que les projets sont axés recherche et développement. Cela a éveillé en moi un attrait pour la recherche et m'incite à continuer mes études dans le monde de la recherche par la réalisation d'une thèse à la suite de ce projet de fin d'études.