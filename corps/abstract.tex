The implementation of a simulator is an essential part of a robotics project. Indeed, it reduces development time and costs by testing the robot in a simulated environment. It allows to place the robots in controlled situations, in a reproducible way and without risk of damage or material failure during tests.

This final year project concerns the implementation of a simulator for \forssea{} \gls{ROV}s. It must allow to simulate the behavior of their robots in a simulated underwater environment to validate the software implementation of new features without having to conduct testing sessions in marine environments which are costly in time and resources. These are still essential to validate a new version of the robot.

First, we will propose a formalism to simulate the marine world by describing the waves, the current, but also elements that will influence the behavior of robots, such as tethers. Then we will see how to simulate the robots by finely simulating each sub-component, which are mostly used in the two \gls{ROV}s developed at \forssea{} \argos{} and \atoll{}. Finally we will be able to analyze the results produced by the simulator and we will see that they are close to the data collected during tests with the \gls{ROV}s.
