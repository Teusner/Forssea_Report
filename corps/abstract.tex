\section*{Résumé}
	L'implémentation d'un simulateur est un aspect important de la réalisation d'un projet robotique. En effet, il permet de gagner du temps de développement, mais aussi d'en réduire les coûts en testant le robot dans un environnement simulé. Il permet de placer les robots dans des situations contrôlées, de manière reproductible et sans risque de casse ou de panne matérielle lors d'essais.

	Ce projet de fin d'étude concerne la réalisation d'un simulateur pour les \gls{ROV}s de \textit{Forssea Robotics}. Il doit permettre de simuler le comportement de leur robots dans un environnement sous-marin simulé afin de tester l'implémentation logicielle des nouvelles fonctionnalités sans avoir à organiser des campagnes d'essais en milieu naturels qui sont coûteux en temps et en ressources. Ceux-ci reste tout de même indispensables pour valider une nouvelle version du robot.

\section*{Mots clés}
Robotique, Simulation, ROV, Gazebo, ROS

\selectlanguage{english}
\section*{Abstract}

	The implementation of a simulator is an essential part of a robotics project. Indeed, it reduces development time and costs by testing the robot in a simulated environment. It allows to place the robots in controlled situations, in a reproducible way and without risk of damage or material failure during tests.

	This final year project concerns the implementation of a simulator for \textit{Forssea Robotics} \gls{ROV}s. It must allow to simulate the behavior of their robots in a simulated underwater environment to validate the software implementation of new features without having to conduct testing sessions in marine environments which are costly in time and resources. These are still essential to validate a new version of the robot.
	
\section*{Keywords}
Robotic, Simulation, ROV, Gazebo, ROS

\selectlanguage{french}

\section*{Remerciements}

Je remercie Gautier \textsc{Dreyfus} pour m'avoir offert l'opprtunité de réaliser un contrat de professionnalisation d'un an avec la start-up \textit{Forssea Robotics} tout en finissant ma formation au sein de l'\gls{ENSTAB}.

Je remercie David \textsc{Barral} et mon tuteur de stage Alaa \textsc{El Jawad} pour m'avoir accueilli dans l'équipe robotique de \textit{Forssea Robotics} et encadré tout au long de cette année. Ils ont su me guider, me conseiller et me former au développement robotique dans le monde de l'entreprise.

Je remercie aussi Auguste \textsc{Bourgois}, Mathieu \textsc{Mege} et Ian \textsc{McElroy} pour le soutien technique qu'ils ont pu m'apporter au cours de ce stage, notamment sur des travaux de débogage de code et de relecture de rapports.