\section*{Résumé}
	L'implémentation d'un simulateur est un aspect important de la réalisation d'un projet robotique. En effet, il permet de gagner du temps de développement, mais aussi d'en réduire les coûts en testant le robot dans un environnement simulé. Il permet de placer les robots dans des situations contrôlées, de manière reproductible et sans risque de casse ou de panne matérielle lors d'essais.
	Ce projet de fin d'étude concerne la réalisation d'un simulateur pour les \gls{ROV}s de \textit{Forssea Robotics}. Il doit permettre de simuler le comportement de leur robots dans un environnement sous-marin simulé afin de tester la partie navigation 

\section*{Mots clés}
Robotique, Simulation, ROV, Gazebo, ROS

\selectlanguage{english}
\section*{Abstract}
	In the context of their profession, engineers have to master basic
	techniques for writing documents.
	
	This guide provides the fundamental elements for writing a project
	report. It is intended for students undertaking projects during their
	studies at \gls{ENSTAB} and it gives an overview of the basics in
	relation to the nature and components of a report, with formatting
	guidelines and typographic rules.
	
\section*{Keywords}
Robotic, Simulation, ROV, Gazebo, ROS

\selectlanguage{french}

\section*{Remerciements}

Je remercie \textsc{Gautier Dreyfus} pour m'avoir offert l'opprtunité de réaliser un contrat de professionnalisation avec \textit{Forssea robotics} tout en finissant ma formation au sein de l'\gls{ENSTAB}.

Je remercie \textsc{David Barral} et mon maître de stage \textsc{Alaa El Jawad} pour m'avoir accueilli dans l'équipe robotique de \textit{Forssea Robotics} et encadré tout au long de cette année.

Je remercie aussi \textsc{Auguste Bourgois} et \textsc{Ian McElroy} pour leur soutien technique qu'ils ont pu m'apporter au cours de ce stage.